\documentclass[a4paper]{article}
\usepackage[utf8x]{inputenc}
\usepackage[brazil]{babel}
\usepackage[T1]{fontenc}
\usepackage{graphicx}

\begin{document}

\begin{titlepage}
 \vfill
  \begin{center}
   {\large \textbf{CENTRO UNIVERSITÁRIO SENAC}} \\
   {\large \textbf{BACHARELADO EM CIÊNCIA DA COMPUTAÇÃO}} \\[4cm]
   
   {\large \textbf{GABRIEL VIEIRA FIGUEIREDO TOMAZ}}\\
   {\large \textbf{TALES CARLOS DE PÁDUA}}\\
   {\large \textbf{VINICIUS DE CARVALHO}}\\[4cm]


   {\Large Mini Games com Visão Computacional}\\[4cm]

\vspace{2cm}
\large \textbf{SÃO PAULO}

\large \textbf{MAIO DE 2014}
\end{center}
\end{titlepage}

\break

\begin{titlepage}
 \vfill
  \begin{center}
   {\large \textbf{CENTRO UNIVERSITÁRIO SENAC}} \\
   {\large \textbf{BACHARELADO EM CIÊNCIA DA COMPUTAÇÃO}} \\[4cm]

   {\large \textbf{Gabriel Vieira Figueiredo Tomaz, Tales Carlos de Pádua, Vinicius de Carvalho.}}\\ [1cm]
  
   {\large \textbf{vieira\_frifri@hotmail.com, talescpadua@gmail.com, carvalho.v@outlook.com.}}\\ [3cm]
   
 
   {\Large Mini Games com Visão Computacional}\\[2cm]

   \hspace{.45\textwidth} 
   \begin{minipage}{.5\textwidth}
   \large "Pequenos jogos eletrônicos utilizando conhecimentos de Visão Computacional apresentados para a conclusão da disciplina Projeto Interativo III, do bacharelado em Ciência da Computação, Centro Universitário Senac."\\[0.5cm]
			Sob orientação do Prof.º: Marcelo Hashimoto
  \end{minipage}
  \vfill

\vspace{1cm}
\large \textbf{SÃO PAULO}

\large \textbf{MAIO DE 2014}
\end{center}
\end{titlepage}


\break

\begin{center}
\textbf{\Large Resumo}
\end{center}

Conforme proposto na disciplina de Projeto Interativo III, a partir do estudo de algoritmos relacionados à visão computacional foram desenvolvidos pequenos jogos eletrônicos (mini games) em linguagem C usando a interface gráfica provida pela biblioteca Allegro 5 e uma interface de acesso à câmeras de vídeo provida pela biblioteca OpenCV, de modo que a visão computacional oferecesse não apenas uma opção de controle para o jogador, mas sim um diferencial na experiência e imersão do usuário ao vivenciar os mini games.\\

Palavras-chave: jogos eletrônicos, visão computacional, Allegro 5.\\	

\break

\begin{center}
\textbf{\Large Abstract}
\end{center}

As proposed by the Interactive Project III discipline, from the study of algorithms related to computer vision were originated little eletronic games (mini games) in C language utilizing a graphical interface provided by Allegro 5 library and a web cam access interface provided by OpenCV library, in order to make computer vision offer not only another controller option for the player, but a different experience and imersion for the user while playing the mini games.\\

Keywords: eletronic games, computer vision, Allegro 5.\\

\break


\begin{center}
\textbf{\Large Introdução}
\end{center}

A visão computacional é uma ciência e tecnologia voltada a lidar com a forma como as máquinas enxergam o mundo ao seu redor. As informações captadas por meio de sensores (como scanners, câmeras de vídeo, etc.) podem ser modeladas de diversas formas a fim de suprir necessidades que permeiam desde ramos diretamente ligados à tecnologia de informação (como robótica e áreas de automação tecnológica) até os que se utilizam da tecnologia para dadas outras necessidades, como ciências ambientais, medicina e outros.\\

Com o objetivo de dar um passo inicial para dentro da visão computacional, este trabalho visa utilizar técnicas e algoritmos da mesma aliada à captação de imagens por câmera de vídeo para a produção de jogos simples, mas que mantenham a jogabilidade focada no poder da visão computacional, de modo que a experiência do jogador, ao invés de ser restringida pela interface proposta, se torne um diferencial por conta deste quesito.

\break

\begin{center}
\textbf{\Large Revisão da Literatura}
\end{center}

\%\% TODO referenciar trabalhos utilizados como base para nossos jogos, etc.\\

\break


\begin{center}
\textbf{\Large Desenvolvimento}
\end{center}

\begin{flushleft}
\textbf{\large Ponto de Partida}
\end{flushleft}

Visto que este trabalho foi o primeiro contato formal com a visão computacional por parte do grupo, a estatégia adotada desenvolver os mini games foi uma via de mão-dupla passando por um \textit{brainstorm} de jogos existentes até quais algoritmos poderiam modelar uma interface de controle aceitável para os mesmos e fazendo o caminho de volta, onde eram estudados algoritmos existentes e se imaginava o que era possível, em termos de jogos, produzir a partir deles.

A partir desta metodologia aliada à orientação e pesquisa, surgiram as tecnicas e algoritmos a seguir e a consequente combinação dos mesmos para elaboração dos jogos. 

\begin{flushleft}
\textbf{\large Distância Euclidiana}
\end{flushleft}
\%\% TODO: distância euclidiana

\begin{flushleft}
\textbf{\large Detecção de Cor}
\end{flushleft}
\%\% TODO: detecção de cor

\begin{flushleft}
\textbf{\large Escala de Cinza}
\end{flushleft}
\%\% TODO: grey scale

\begin{flushleft}
\textbf{\large Binarização}
\end{flushleft}
\%\% TODO: binarização

\begin{flushleft}
\textbf{\large Filtro Sobel}
\end{flushleft}
\%\% TODO: sobel operator para detecção de bordas

\begin{flushleft}
\textbf{\large Filtro Gaussiano}
\end{flushleft}
\%\% TODO: distância euclidiana

%\begin{figure}[!htb]
%\centering
%\includegraphics{tela_do_jogo.png}
%\caption{Figura 1 ? Tela do jogo com as cartas para a implementação dos algoritmos (Fonte: Jogo, Código de Honra criado pelo grupo ? Retirada 18/11/2013)}
%\label{Rotulo}
%\end{figure}

%\begin{figure}[!htb]
%\centering
%\includegraphics{git.png}
%\caption{Figura 2 ? Repositório do jogo criado para auxiliar o compartilhamento de dados via Web (Fonte: Github - https://github.com/tgl-dogg/BCC_PI2_CDH ? Retirada 20/11/2013)}
%\label{Rotulo}
%\end{figure}

\break

\begin{center}
\textbf{\Large Resultados}
\end{center}

\%\% TODO o que conseguimos produzir com os algoritmos citados no desenvolvimento, e descrever por que funcionou fazer os jogos desta forma. 

\break

\begin{center}
\textbf{\Large Considerações Finais}
\end{center}

\%\% TODO conclusões sobre o nosso trabalho.\\

Ex: Flappy Bino: estudamos X algoritmo e Y técnica mas não deu certo por motivo A.\\

Jogo do Mexe-Mexe: com o X algoritmo do Flappy Bino conseguimos fazer o jogo com sucesso por motivo B.

\break

\begin{center}
\textbf{\Large Referências Bibliográficas}
\end{center}

\%\% TODO formatar referências em ABNT \\
- Livros do nosso Google Drive sobre Visão Computacional \\
- Livro de Álgebra Linear da aula da Dani com a distância euclidiana \\
- Tutoriais do OpenCV \\

%% um pouco de introdução à computer vision eu tirei daqui:
%% http://en.wikipedia.org/wiki/Computer_vision

\break

\end{document}